{\footnotesize
  \lstset{language=C}
  \begin{lstlisting}[frame=shadowbox,breaklines]
    definitions:
    	OPTIMIZE_EVERY	// every time this number of steps has been performed, an optimization sequence is made
        MIN_ANGLE_DIST	// angles threshold for observations association
        SECOND_MIN_ANGLE_DIST	// P x MIN_ANGLE_DIST, P>=1
    
    global variables: 
    	landmarks_list	// list where landmarks are inserted
        graph	// the graph to be used for optimization
        
    // This function manages the algorithm iterations.
    // 'steps' is the sequence of steps, each with the relative transformation with respect to the previous step and a set of observations.
    algorithm_launcher(steps)
    {
      // 'previous_observations' and 'propagate_observations' are lists used to keep track step after step of the observed landmarks
      previous_observations = new empty Landmarks list;
      propagate_observations = new empty Landmarks list;
      
      int loop_count = 0;	// increases at every step, is set to 0 when optimization is performed
      
      for each step
        loop_count ++;
      	current_position = compute the new position appending the new step to the previous one;
        current_position.observations = step.observations;
        
        // the observations propagated from the previous step are moved to the specific list
        previous_observations = propagate_observations;
        clear propagate_observations;
        
        propagate_observations = tryToUnderstand(current_position, previous_observations);
        
        if loop_count > OPTIMIZE_EVERY
        	loop_count = 0;
                populate the graph with new landmarks nodes and edges;
                graph.optimize();
        
      // end of 'for each step'
      
      // last optimization
      populate the graph with new landmarks nodes and edges;
      graph.optimize();
      
      // expectation maximization
      in case it is needed/required_by_the_user
      	steps = compute relative transformations from the optimized positions;
        algorithm_launcher(steps)	// rerun the algorithm
    }
    
    
    // this is the actual data association function
    // returns a Landmark list containing the tracks to be propagated to the next step
    // REMEMBER that 'tracks' are just 'landmarks', hence the term is often used interchangeably
    tryToUnderstand(current_pose, previous_observations)
    {
      // prepare structures
      list<Landmark> propagate;	// list used to propagate tracks to the next step.
      
      double expected[previous_observations.size()];	// an array of doubles with a value for each of the previous observations
      
      int associations[current_pose.observations.size()]	// an array with as many integers as the measurements in the current pose.
      
      // populate 'expected' with the angles where we expect to see the landmarks that have been propagated from the previous step
      for each landmark 'lmark' in previous_observations
      	// first case
      	if the landmark has an estimated position
          expected.add(backprojection(lmark, current_pose));
        else  // second case
          prev_measure = extract last measured angle for the landmark;
          expected.add(prev_measure - relative_rot);	// relative_rot is the rotation performed from the previous step
      
      // at this point, expected contains the angles to be compared with new observations
      
      // try to associate new observations to previous tracks
      for each observation 'obs' in current_pose.observations
        
        // compute the angle distances from the 'expected' angles 
        double distances[previous_observations.size()];
        distances <- computeDistances(obs.bearing, expected);
        // 'distances' now contains the ``distance'' of the current observation from each of the expected values
        
        // 'closer' and 'second_closer' will be used to identify the two most likely tracks for this observation to be associated
        int closer;
        int second_closer;
        
        closer =  index of min(distances);
        second_closer = index of min_i(distances, i!=closer);
        
        if distances[closer] < MIN_ANGLE_DIST	// the closer expected angle is close enough
        	
        	//first case
                if distances[second_closer] > SECOND_MIN_ANGLE_DIST
                	// the second closer angle is far enough: no ambiguity.
                        // store information that this angle should be associated to the landmark that generated the considered expected value
                        associations[obs] = closer;
                        // note that the observation is not added to the track yet, because there ould still rise ambiguity with 
                else
                	// the second closer angle is lower than its threshold: there is ambiguity
                        associations[obs] = -1	// just an invalid value.
                        
        else	// distances[closer] > MIN_ANGLE_DIST: this new observation is far from all the existing tracks
        	associations[obs] = -1	// just an invalid value.
      
      // end of 'for each observation...'
      
      // at this point associations[i] contains, which is the track to which the new i-th observation should be associated. If there are no associable tracks or there are more than one, this value is -1 (invalid)
      
      // next step: determine another kind of ambiguity: when two new observations ask to be associated to the same existing track
      
      bool has_new_obs[previous_observations.size()];
      // a boolean for each track
      // has_new_obs[i] will be set to true if at least one new observation is associable to the i-th track
      
      bool ambiguous[previous_observations.size()];
      // a boolean for each track
      // ambiguous[i] will be set to true if more than one observation is associable to the i-th track
      
      // initialize the two arrays to all false values
      has_new_obs <- false;
      ambiguous <- false;
      
      // look for ambiguities
      for each observation 'obs' in current_pose.observations
      	int value = associations[obs];	// value is the index of the track to be associated
        
        if value != -1
        	if has_new_obs[value] == false
                	has_new_obs[value] = true
                else
                	ambiguous[value] = true
                        
      // at this point ambiguous[i] is true if the i-th track has two or more potential new observations
      // add non-ambiguous new observations to the corresponding track, and create new tracks for the newly observed unassociated observations
      
      for each observation 'obs' in current_pose.observations
      	track_index = associations[obs];
      	if track_index != -1
        	if ambiguous[track_index] == false	// obs can be tailed to the track
        	        selected_track = previous_observations[track_index];
                        selected_track.addObservation(obs);
                	re-estimate landmark position;
                        check if the landmark should be set to 'confirmed', and in case set it;
                        propagate.add(selected_track);
                else
                	// obs can't be tailed to the track, because there is ambiguity
        else
        	// this observation has been considered a new track
                Landmark lm = new Landmark();
                lm.addObservation(obs);
                landmarks_list.add(lm);	// add the landmark to the overall landmarks set
                propagate.add(lm);	// add the landmark to the tracks to be considered in the next step
    
      
      // now re-scan all the tracks from previous step, and if needed prune them (kill the landmark).
      for each track 'tr' in previous_observations
      	if has_new_obs[tr] == false
        	// we expected to see this landmark, but we didn't so we have two cases:
                if tr.confirmed == true
                	// we trust this landmark
                else
                	// this track is too young
                        landmarks_list.remove(tr);	// delete the landmark from the overall landmarks set
        
        else
        	// some new observations were close to this track
                if ambiguous[tr]
                	// this track was not enhanched, because two or more observations were close to it
                        // check it's reliability
                        if tr.confirmed == true
                        	// it is the case to keep tracking this valuable landmark
                                propagate.add(tr);
                        else
                        	// this 'young' track is already giving problems... just remove it
                                landmarks_list.remove(tr);
                
      
    return propagate;
    }
  \end{lstlisting}
}

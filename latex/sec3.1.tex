\section{Simulator in Matlab}\label{sec:matlab_simulator}

This Matlab odometry simulator was realized in order to consider the uncertainties on robot perceptions originated from physical process like wheels slippage,
or sensor noise.
The simulator provides a GUI that allows the user to navigate in a map filled with randomly-placed landmarks, and visualize in real-time both the real configurations sequence 
and the noised configurations sequence.
Finally, a text file is produced, associating a sequence of robot states to the landmark perceptions from these states.

\subsection{Representation of the configuration}
A robot configuration $C$ is rapresented by the mean of a 2D homogeneous matrix.

\[ 
\textbf{C}(k) = 
\left( \begin{array}{c:c}
  \textbf{R}(k)  & \textbf{t}(k) \\ \hdashline
  0 & 1  \\
\end{array} \right)
\]
with $\textbf{t} = [x,y]^T$.\\

The matrix also rapresents univocally a certain state of the robot $[x,y,\theta]$.
\\
The robot model used is a planar unicycle, with two control inputs (linear and angular velocity) controllable by human user.
The robot is equipped with a single laser scanner capable of returning the bearing of all landmarks within a certain distance.

The instantaneous linear velocity $v \in \mathbb{R}$ is directed in the $x$ (sagittal) axis of the robot, while angular velocity $\theta \in \mathbb{R}$ is
around the $z$ axis of the workspace (orthogonal to the plane). Of course this values are intended to be deduced from the encorders on the wheels.

Given a certain configuration $\textbf{C}$ and the two instantaneous velocities $v$ and $\theta$, the new configuration is computed as:

\[ 
\textbf{C}(k+1) = 
\left( \begin{array}{c:c}
  %%\textbf{R(k)} * \textbf{H_r}  & \textbf{t(k)} + \textbf{\tilde{t}} \\ \hdashline
  \textbf{R}(k) \cdot \textbf{H}_r & \textbf{t}(k) + \tilde{\textbf{t}} \\ \hdashline
  %%\textbf{R(k)}  & \textbf{t(k)} \\ \hdashline
  0 & 1  \\
\end{array} \right)
\]

\[ 
\tilde{\textbf{t}} = [\tilde{t_x},\tilde{t_y}]^T = \textbf{R}(k) \cdot [v,0]^T
\]

\[ 
\textbf{H}_r =
\left( \begin{array}{cc}
  cos(\theta)  & -sin(\theta)\\ 
  sin(\theta)  & cos(\theta) \\
\end{array} \right)
\]

This model is equivalent to the unicycle in [1] (cita libro robotica).
Note that the integration method used is the \textit{approximated Euler}, since we are considering that the $\theta$ remains constant during the integration step,
and changes suddenly at the next step, instead of changing continuously as it would be in reality.

\subsection{Configurations storing}
The \textit{distance} $d$ between two configuration matrices $M_1$, $M_2$ rapresenting respectively states $s1 = [x_1,y_1,\theta_1]$, ~ $s2 = [x_2,y_2,\theta_2]$
is defined as
\[ 
d = \alpha \cdot d_{\theta} + d_p
\]
where $d_{\theta}$ is the absolute value of the the (circular) distance between angles $\theta_1$,$\theta_2$,
and $d_p$ is the norm of the vector $[x_1 - x_2,y_1 - y_2]$.
The parameter $\alpha$, usually $> 1$, can balance the importance of the rotation distance respect the linear one.\\
While user navigation proceeds the simulator stores in two separete list (real and noised) all the subsequent configurations $\textbf{C}$.
%%These two list are used to provide the user in real time the comparison between real and noised configurations.
(IMAGE)
In a certain time instant, both current noised and real configurations are appended to their lists if and only if the \textit{distance} between the current (noised) configuration 
and the last saved (noised) configuration exceeds a certain threeshold.
\\
\subsection{Landmark readings}
Every time that a configuration is appended to the list, a new \textit{landmark reading} is created.
A \textit{landmark reading} is composed by two rows:\\
\[ 
odomPose, ~~ <ID>, ~~ n_X, ~ n_Y, ~  n_{\theta}, ~  t_X, ~  t_Y, ~ t_{\theta}
\]
\[
bearing, ~~ b_1, ~ ... ~ b_n, ~ IDb_1, ~ ... ~ IDb_n, 
\]

where:
\textit{$<ID>$} is the number of reading, $[n_X, n_Y, n_{\theta}]$ is the noised state of the robot, while $[t_X, t_Y, t_{\theta}]$ the real one.
The $n$ values $ b_1 ... b_n$ gives the bearing angle of the $n$ landmarks visible from $[t_X t_Y t_{\theta}]$. Each landmark is then associated to his univocal ID 
$<IDb_1> ... <IDb_n>$.\\
At the end of the navigation an output file consisting in a sequence of \textit{landmark readings} is produced from the simulator.

\subsection{Odometry and sensors noise}
The error on the odometry is modeled as a zero-centered white noise with gaussian distribution, that influences independently the two velocities $v$ and $\theta$.
The standard deviation of the distribution has a constant component and an additional component that is directly proportional to the norm of velocity $v$ or $\theta$.
In this way is modelled the fact that usually in reality we tends to have higher odometric noise with higher robot wheels velocities.
\\
The error on the sensor is modelled as a zero-centered white noise with gaussian distribution, that is directly summed to the bearing readings.

Note that the Odometry error is added while navigating, and influences the integration done on the robot. Conversely the error on sensor is added off-line, and it is indipendent 



